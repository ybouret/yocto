\documentclass[aps]{revtex4}
\usepackage{graphicx}
\usepackage{amssymb,amsfonts,amsmath,amsthm}
\usepackage{chemarr}
\usepackage{bm}
\usepackage{bbm}
\usepackage{pslatex}
\usepackage{mathptmx}
\usepackage{xfrac}

\newcommand{\mymat}[1]{\bm{#1}}
\newcommand{\mytrn}[1]{~^{\mathsf{T}}{#1}}
\newcommand{\mygrad}{\vec{\nabla}}

\begin{document}
\title{Chemical Solutions}

\section{Description}
Let us assume that we have $A_1,\ldots,A_M$ chemical species coupled by
$N$ equilibria such that
\begin{equation}
	\forall i \in [1;N], \;\; \sum_{j=1}^{M} \nu_{i,j} C_i = 0, \;\; K_i(t) = \prod_{i=1}^{M} C_i^{\nu_{i,j}}.
\end{equation}
We remove the singularities by assuming that the equilibria are met when
\begin{equation}
	\forall i \in [1;N], \;\; \Gamma_i(t,\vec{C}) = K_i(t) \prod_{\nu_{i,j}<0}  C_i^{-\nu_{i,j}} -  \prod_{\nu_{i,j}>0} C_i^{\nu_{i,j}} 
\end{equation}
or
\begin{equation}
	\vec{\Gamma}(t,\vec{C}) = \vec{0}.
\end{equation}
We also naturally have the topology matrix $\mymat{\nu}$, such that
any chemical shift is defined by a  chemical extent vector $\vec{\xi}$ and
produces a shift
$$
	\mytrn{\mymat{\nu}}\vec{\xi}
$$
We note $\mymat{\Phi}$ the Jacobian of $\vec{\Gamma}$.

\section{Balancing Concentrations}

Let us assume that we define a continuously derivable objective function $\mathcal{E}\left(\vec{C}\right)$ which is minimal (with value 0) when all active species are positive. Whenever one of those concentrations is negative, we want to find an excess
$\delta\vec{C}$ such that
\begin{equation}
	\mathcal{E}\left(\vec{C}+\delta\vec{C}\right) \leq 0
\end{equation}

Let $\vec{\beta}$ be the steepest descent direction of $\mathcal{E}$,
\begin{equation}
	\vec{\beta} = -\partial_{\vec{C}} \mathcal{E}.
\end{equation}
Since
\begin{equation}
	\mathcal{E}
	\left(\vec{C}+\mytrn{\mymat{\nu}}\vec{\xi}\right) 
	\simeq \mathcal{E}\left(\vec{C}\right) - \vec{\beta} \cdot \mytrn{\mymat{\nu}}\vec{\xi}
	= \mathcal{E}\left(\vec{C}\right) - \mymat{\nu}\vec{\beta}\cdot\vec{\xi},
\end{equation}
the steepest descent direction w.r.t the extent is $\mymat{\nu}\vec{\beta}$, which turns into
a steepest concentration descent 
\begin{equation}
	\vec{g} = \mytrn{\mymat{\nu}} \mymat{\nu} \, \vec{\beta}
\end{equation}
%so that we want to find an extent $\vec{\xi}$ which minimises $\mathcal{F}(\alpha)=\mathcal{E}(\vec{C} + \alpha

We may use
\begin{equation}
\mathcal{E}\left(\vec{C}\right) = \sum_{j} 
\left\lbrace
\begin{array}{ll}
\frac{1}{2} C_j^2 & \text{if species $\#j$ is  active  and $C_j<0$}  \\
0 & \text{otherwise}\\
\end{array}
\right. 
\end{equation}
so that
\begin{equation}
	\vec{\beta} =
	\begin{bmatrix}
	\vdots\\
	-C_j \; (\text{if species $\#j$ is  active  and $C_j<0$})\\
	\vdots\\
	\end{bmatrix}
\end{equation}

Hence we can write a modified conjugated gradient algorithm, which makes sens only if there is
an invalid concentration

\begin{enumerate}
\item Initialize $\vec{g}$, check that $g_2=\vert\vec{g}\vert^2>0$. Set $\vec{h}=\vec{g}$.
\item Find the minimum, w.r.t $\alpha$ of $\mathcal{E}\left(\vec{C}+\alpha \vec{h}\right)$. 
\item Update $\vec{C}$ to $\vec{C}+\alpha\vec{h}$. If the minimum is zero, terminate the algorithm.
\item Compute the temporary steepest descent $\vec{b}$, check that $b_2=\vert\vec{b}\vert^2>0$.
\item Compute $\gamma=\dfrac{\left(\vec{b}-\vec{g}\right)\cdot \vec{b}}{g_2}$.
\item Update $\vec{g} = \vec{b}$
\item Update $\vec{h} = \vec{g} + \gamma \vec{h}$
\item go to 2
\end{enumerate}

\section{Normalising}
A chemical system must be balanced before normalisation.
Then, at a given time $t$, we assume that we start from
$\vec{\Gamma}\left(\vec{C}\right)\not=\vec{0}$  and that we look for the chemical
extent $\vec{\xi}$ so that
\begin{equation}
	\vec{\Gamma}\left(\vec{C}+\mytrn{\mymat{\nu}}\vec{\xi}\right) = \vec{0}
\end{equation}
We approximate the extent using the first order expansion
\begin{equation}
	\vec{0} = \vec{\Gamma}\left(\vec{C}\right) + \underbrace{\left(\mymat{\Phi}\mytrn{\mymat{\nu}}\right)}_{\displaystyle \mymat{\chi}} 
	\vec{\xi}
\end{equation}	
\begin{enumerate}
\item Compute $\vec{\Gamma}$ and $\mymat{\chi}$
\item Compute $\vec{\xi}=-\mymat{\chi}^{-1} \vec{\Gamma}$
\item Update  $\vec{C}$ to $\vec{C}+\mytrn{\mymat{\nu}}\vec{\xi}$
\item Balance !
\item go to 1 until convergence.
\end{enumerate}

\section{Using in an ODE}

\section{Booting with linear constraints}
\subsection{Description}
Let us assume that we have $N_c$ linear constraints (electroneutrality, osmolarity, mass concentration) with $Nc+N=M$
and we can write that a valid concentration must satisfy, using a $N_c\times M$ matrix $\mymat{P}$. 
\begin{equation}
	\mymat{P} \vec{C} = \vec{\Lambda} \text{ and } \vec{\Gamma}\left(\vec{C}\right) = \vec{0}.
\end{equation}
Obviously, if we take $\mymat{Q}\in\mymat{P}^{\perp}$, we have a set of vectors $\vec{U}\in\mathbb{R}^{N_c}$ and $\vec{V}\in\mathbb{N}$ such that
\begin{equation}
	\vec{C} = \mytrn{\mymat{P}} \vec{U} + \mytrn{\mymat{Q}} \vec{V}
\end{equation}
We must find a way to compute $\vec{U}$ and $\vec{V}$ while preserving the validity of $\vec{C}$.

\subsection{Algorithm}
We need a globally decreasing strategy...\\
Let us assume that we start from $\vec{V}_n$ defining $\vec{C}_n$.
We also define $\mathcal{C}$ the operator that set invalid concentrations to $0$, and $\gamma$ an
objective function of $\vec{\Gamma}$.

We start from a VALID $\vec{C}_n$, with a computed $\vec{\Gamma}_n$ and $\mymat{\Phi}_n$, and
the value $\gamma_n$.

\begin{enumerate}
\item Compute the linear 
	$$\delta\vec{U}_n = \left(\mymat{P}\mytrn{\mymat{P}}\right)^{-1} \left( \vec{\Lambda} - \mymat{P} \vec{C}_n \right)$$

\item Compute the non-linear
	$$\delta\vec{V}_n = - \left(\mymat{\Phi}_n\mytrn{\mymat{Q}}\right)^{-1}\left(\vec{\Gamma}_n+\mymat{\Phi}_n\underbrace{\mytrn{\mymat{P}}\cdot\delta\vec{U}_n}_{\delta\vec{C}_n'}\right)$$

\item Compute the descent direction 
$$
	\delta\vec{C}_n = \underbrace{\mytrn{\mymat{P}} \delta \vec{U}_n}_{\delta\vec{C}_n'} + \mytrn{\mymat{Q}} \delta\vec{V}_n
$$

\item Find $\alpha\geq0$ which minimises 
$$
	\gamma_{n+1} = \gamma\left( \mathcal{C}\left(\vec{C}_n + \alpha \delta\vec{C}_n\right)\right)
$$

\item if $\gamma_{n+1}\geq\gamma_n$, stop and use $\vec{C}_n$ 
\item otherwise, update $\vec{C}_n$, $\gamma_n$.
\end{enumerate}

The the validity of the result must be checked...


\end{document}




