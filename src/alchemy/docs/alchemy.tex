\documentclass[aps]{revtex4}
\usepackage{graphicx}
\usepackage{amssymb,amsfonts,amsmath,amsthm}
\usepackage{chemarr}
\usepackage{bm}
\usepackage{bbm}
\usepackage{pslatex}
\usepackage{mathptmx}
\usepackage{xfrac}

\newcommand{\mymat}[1]{\bm{#1}}
\newcommand{\mytrn}[1]{~^{\mathsf{T}}{#1}}
\newcommand{\mygrad}{\vec{\nabla}}

\begin{document}
\title{Chemical Solutions}

\section{Description}
Let us assume that we have $A_1,\ldots,A_M$ chemical species coupled by
$N$ equilibria such that
\begin{equation}
	\forall i \in [1;N], \;\; \sum_{j=1}^{M} \nu_{i,j} C_i = 0, \;\; K_i(t) = \prod_{i=1}^{M} C_i^{\nu_{i,j}}.
\end{equation}
We remove the singularities by assuming that the equilibria are met when
\begin{equation}
	\forall i \in [1;N], \;\; \Gamma_i(t,\vec{C}) = K_i(t) \prod_{\nu_{i,j}<0}  C_i^{-\nu_{i,j}} -  \prod_{\nu_{i,j}>0} C_i^{\nu_{i,j}} 
\end{equation}
or
\begin{equation}
	\vec{\Gamma}(t,\vec{C}) = \vec{0}.
\end{equation}
We also naturally have the topology matrix $\mymat{\nu}$, such that
any chemical shift is defined by a  chemical extent vector $\vec{\xi}$ and
produces a shift
$$
	\mytrn{\mymat{\nu}}\vec{\xi}
$$

\section{Balancing Concentrations}
\subsection{Method}

Let us assume that we define an objective function $\mathcal{E}\left(\vec{C}\right)$ which is minimal with value 0 when all active species are positive. Whenever one of those concentrations is negative, we want to find and extent $\vec{\xi}$ which optimises 
\begin{equation}
	\mathcal{E}\left(\vec{C}+\mytrn{\mymat{\nu}}\vec{\xi}\right)
\end{equation}
The descent direction w.r.t. $\vec{\xi}$ is
\begin{equation}
	-\partial_{\vec{\xi}} \mathcal{E} = -\mymat{\nu} \mygrad \mathcal{E} = \mymat{\nu} \vec{\beta}
\end{equation}
with
\begin{equation}
	\vec{\beta} = -\mygrad \mathcal{E}.
\end{equation}
Accordingly, 
\begin{equation}
\exists \alpha \geq 0,\text{ such that }\delta\vec{C}_\alpha 
= \alpha \mytrn{\mymat{\nu}} \mymat{\nu} \vec{\beta}
= \alpha \vec{\eta} \text{ will decrease } \mathcal{E}\left( \vec{C}+\delta\vec{C}_\alpha \right)
\end{equation}

\subsection{Function}
The simplest function is
\begin{equation}
	\mathcal{E}\left(\vec{C}\right) = \sum_j
	\left\lbrace
	\begin{array}{cl}
	-C_j & \text{if species } j \text{ is active and } C_j<0\\
	0    & \text{otherwise}\\
	\end{array}
	\right.
\end{equation}

Accordingly
\begin{equation}
	\vec{\beta} = 
\end{equation}



\end{document}

