\documentclass[aps]{revtex4}
\usepackage{graphicx}
\usepackage{amssymb,amsfonts,amsmath,amsthm}
\usepackage{chemarr}
\usepackage{bm}
\usepackage{bbm}
\usepackage{pslatex}
\usepackage{mathptmx}
\usepackage{xfrac}

\newcommand{\mymat}[1]{\bm{#1}}
\newcommand{\mytrn}[1]{~^{\mathsf{T}}{#1}}
\newcommand{\mygrad}{\vec{\nabla}}

\begin{document}
\title{Chemical Solutions}

\section{Description}
Let us assume that we have $A_1,\ldots,A_M$ chemical species coupled by
$N$ equilibria such that
\begin{equation}
	\forall i \in [1;N], \;\; \sum_{j=1}^{M} \nu_{i,j} C_i = 0, \;\; K_i(t) = \prod_{i=1}^{M} C_i^{\nu_{i,j}}.
\end{equation}
We remove the singularities by assuming that the equilibria are met when
\begin{equation}
	\forall i \in [1;N], \;\; \Gamma_i(t,\vec{C}) = K_i(t) \prod_{\nu_{i,j}<0}  C_i^{-\nu_{i,j}} -  \prod_{\nu_{i,j}>0} C_i^{\nu_{i,j}} 
\end{equation}
or
\begin{equation}
	\vec{\Gamma}(t,\vec{C}) = \vec{0}.
\end{equation}
We also naturally have the topology matrix $\mymat{\nu}$, such that
any chemical shift is defined by a  chemical extent vector $\vec{\xi}$ and
produces a shift
$$
	\mytrn{\mymat{\nu}}\vec{\xi}
$$

\section{Balancing Concentrations}
\subsection{Method}

Let us assume that we define an objective function $\mathcal{E}\left(\vec{C}\right)$ which is minimal (with value 0) when all active species are positive. Whenever one of those concentrations is negative, we want to find an excess
$\delta\vec{C}$ such that
\begin{equation}
	\mathcal{E}\left(\vec{C}+\delta\vec{C}\right).
\end{equation}
Locally, we have
\begin{equation}
	\mathcal{E}\left(\vec{C}+\mytrn{\mymat{\nu}}\vec{\xi}\right) \simeq 
	\mathcal{E}\left(\vec{C}\right) - \vec{\beta}\cdot\delta\vec{C}.
\end{equation}
Since, for matter conservation, we must have $\delta\vec{C} = \mytrn{\mymat{\nu}}\vec{\xi}$, 
a condition is that there exists a positive value $\mu$ such that 
\begin{equation}
	\vec{\xi} \cdot \mymat{\nu} \vec{\beta} = \mu > 0 .
\end{equation}
In particular, this fails is $\vert\mymat{\nu} \vec{\beta}\vert=0$...
We need to choose $\vec{\xi}$. A simple way is to impose a least displacement condition, so
that we want to minimise $\vert\delta\vec{C}\vert$, or $\vert\mytrn{\mymat{\nu}}\vec{\xi}\vert$, or
$\dfrac{1}{2}\vec{\xi}\mymat{\nu}\mytrn{\mymat{\nu}}\cdot\vec{\xi}$.

Hence we want to minimise the Lagrangian
\begin{equation}
	\mathcal{L} = \dfrac{1}{2}\vec{\xi}\mymat{\nu}\mytrn{\mymat{\nu}}\cdot\vec{\xi} 
	- \lambda \left( \vec{\xi} \cdot \mymat{\nu} \vec{\beta} - \alpha\right)
\end{equation}
leading to
\begin{equation}
	\vec{\xi} = \lambda \left(\mymat{\nu}\mytrn{\mymat{\nu}}\right)^{-1} \mymat{\nu} \vec{\beta}
\end{equation}
and
\begin{equation}
	\mu = \lambda \vec{\beta} \mytrn{\mymat{\nu}} \left(\mymat{\nu}\mytrn{\mymat{\nu}}\right)^{-1} \mymat{\nu} \vec{\beta}
\end{equation}

By a change of scaling, there exists $\alpha$ such that
\begin{equation}
	\delta \vec{C} = \alpha \underbrace{\mytrn{\mymat{\nu}} \left(\mymat{\nu}\mytrn{\mymat{\nu}}\right)^{-1} \mymat{\nu}}_{\mymat{W}} \cdot \vec{\beta}
\end{equation}
%Indeed, the determinant of a Gram matrix is always positive, so that we can use the  adjoint matrix.\\
So, once we have computed $\vec{\beta}$, we have
\begin{equation}
	\vec{\eta} = \mymat{W} \vec{\beta}
\end{equation}



\end{document}

