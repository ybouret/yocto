\documentclass[aps]{revtex4}
\usepackage{graphicx}
\usepackage{amssymb,amsfonts,amsmath}
\usepackage{chemarr}
\usepackage{bm}
\usepackage{pslatex}
%\usepackage{mathptmx}
\usepackage{xfrac}

\begin{document}
\title{Regular splitting of quad meshes}
\maketitle

\section{Notation}

Let us assume that we have a quad mesh of dimension $D$,
with layout $\vec{L}$ that we want to dispatch
on $C$ compute engines.\\
We also assume that the compute time per slot is 
$$\tau$$ and
that the communication time per boundary slot is 
$$\alpha.$$
The sequential compute time is
$$
	\tau \left(\prod_{d=1}^{D} L_d\right)=N_0 \tau
$$
For any number of compute engines, we have $P$ ways to
subdivide the layout in $C$ partitions, so that we may test
the partions
$$
	Q_{ij},\;{i=1\ldots P,\;j=1\ldots C},
$$
with the corresponding boundaries $B_{ij}$, where we neglect the local copies.

\subsection{Reference partition}
The reference partition is found by using the partition $Q_r$ corresponding
to a regular split in the widest direction. The number of items to process
is then $N_r = \sup_{j}\left\lbrace \#Q_{rj} \right\rbrace$ 
with the corresponding number of items to transfer $A_r = \#B_{rj_r}$.
We hence assume that choosing $C$ compute engines yields a speed up, so that
$$
	N_r\tau+A_r \alpha \leq N_0 \tau,
$$
meaning that
$$
	\alpha \leq \dfrac{N_0-N_r}{A_r} \tau.
$$

\subsection{Finding out the partition}

For each partition, we have to compute the total compute time
assuming that $\alpha$ is the value corresponding to the reference time, and
see if it is faster.
For any partition $Q_p$, we compute
$$
	N_p = \sup_{j}\left\lbrace ... \right\rbrace
$$

\end{document}

\section{Generic}

We have a 2D box with lengths $L_x,L_x$ and
we want to split the layout with $N$ regions, under the assumption
of fully periodic boundary conditions.\\
 The computation
that we want to carry out depends on both the size of
a region and the time to transfer information between two
regions.\\
We assume that $\alpha$ is the time to transfer the needed data 
between two regions \emph{per unit of length}, and
we neglect the local copy times.
The sequential compute time is $\tau$.
\begin{itemize}
\item If we split along $X$, we have a compute time of
\begin{equation}
	\Theta_x = \dfrac{\tau}{N} + 4\alpha L_y,
\end{equation}
the factor $4$ arising from two send, two receive operations...

\item If we split along $Y$, we have a compute time of
\begin{equation}
	\Theta_y = \dfrac{\tau}{N} + 4\alpha L_x
\end{equation}
\end{itemize}

We remark that
\begin{equation}
	\Theta_x - \Theta_y = 4\alpha\left(L_y-L_x\right),
\end{equation}
so we would always split in the largest direction.\\


If we split along $XY$ with $N=n_x \times n_y$ and $n_x>1,n_y>1$ (otherwise we are linear),
we get the compute time of
\begin{equation}
	\Theta_{xy} = \dfrac{\tau}{N} + 4\alpha\left(\dfrac{L_x}{n_x}+\dfrac{L_y}{n_y}\right).
\end{equation}
Let us assume that, by symmetry, $L_x\geq L_y$, so that we want to evaluate
\begin{equation}
	\delta \Theta = \Theta_{xy}-\Theta_x = 4\alpha\left(\dfrac{L_x}{n_x}+\dfrac{L_y}{n_y}\right) -
	4\alpha L_y.
\end{equation}	
We get
\begin{align}
	\delta \Theta \leq 0 & \Leftrightarrow n_yL_x + n_xL_y - N L_y \leq 0 \\
	& \Leftrightarrow \dfrac{N}{n_x} L_x + n_x L_y - N L_y \leq 0\\
	& \Leftrightarrow N L_x + n_x^2 L_y - n_x N L_y \leq 0\\
	& \Leftrightarrow n_x^2 - N n_x + N \dfrac{L_x}{L_y} \leq 0
\end{align}

The discriminant of the left side is 
\begin{equation}
	\Delta = N^2 - 4 N \dfrac{L_x}{L_y}.
\end{equation}
\begin{itemize}
	\item If 
		\begin{equation}
		\dfrac{L_x}{L_y}\geq \dfrac{N}{4}
		\end{equation}
		ie the shape is too "long" in the $X$ direction, it's always better to split along $X$ only.
	\item Otherwise, it may be possible to choose $n_x$ between the two roots of the previous equation
	and if $n_x>1$, $n_y>1$ and $n_x \times n_y = N$.
\end{itemize}

\section{Finding Out}
We use integer values.
We define the timings relative to the cost of per item computation.
$$ 
	\Theta_0 = L_x \times L_y
$$
Assuming that we gain something by splitting in the largest direction $u$, so that the communication time is proportional
to $L_v$ with a factor $\beta$ such that
$$
	\Theta_u = \dfrac{\Theta_0}{2} + \beta  L_v \leq \Theta_0
$$
we obtain a maximal communication rate
$$
	\beta \leq \dfrac{\Theta_0}{2L_v}.
$$

Let us have $N_{cpu}$ on with we can dispatch the computation.
Then we have
$$
	N_{x} = \min\left(L_x,N_{cpu}\right), \; N_y = \min\left(L_y,N_{cpu}\right)
$$
So we initialise the algorithm with the maximum number of cores $N_u$ in the $u$ direction (with may be less than the the number of processors...).
And we have the initial minimal time
$$
	\Theta_{min} = \dfrac{\Theta_0}{N_u} + \beta L_v =  \dfrac{\Theta_0}{N_u} + \dfrac{\Theta_0}{2} = \dfrac{\left(2N_u+1\right)\Theta_0}{2N_u}.
$$
Hence, we initialize $\Theta_{num}=\left(2N_u+1\right)$ and $\Theta_{den}=2N_u$.
Then
\begin{itemize}
	\item for all $2\leq p_x \leq N_x$
	\begin{itemize}
	\item for all $2\leq p_y \leq N_y$
	\item if $p=p_x \times p_y>N_{cpu}$ then break...
	\item The time per core is
		$$
			\Theta_{xy} = \dfrac{\Theta_0}{p} + \beta\left(\dfrac{L_x}{p_x}+\dfrac{L_y}{p_y}\right)
		$$
		or
		$$
			\Theta_{xy} = \dfrac{2\Theta_0+2\beta\left(p_y L_x + p_x L_y\right)}{2p}
		$$
		and
		$$
			\Theta_{xy} = \dfrac{\Theta_0 \left(2L_v + p_y L_x + p_x L_y\right)}{2pL_v}
		$$
	\item done $p_y$
	\end{itemize}
	\item done $p_x$
\end{itemize}

\end{document}

