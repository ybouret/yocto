\documentclass[aps,onecolumn,11pt]{revtex4}
%\documentclass[11pt]{article}
%\usepackage[cm]{fullpage}
\usepackage{graphicx}
\usepackage{amssymb,amsfonts,amsmath,amsthm}
\usepackage{chemarr}
\usepackage{bm}
\usepackage{pslatex}
\usepackage{mathptmx}
\usepackage{xfrac}
\usepackage{xcolor}
\usepackage{bookman}

\begin{document}

Let us have a regular mesh that we want to divide into a regular partition of domains, using at most $C$ cores.
How do we choose?\\
Let us compute all the acceptable partitions, with $n_g$ ghosts if necessary, and we note this $N$ partitions 
$P_1,\ldots,P_i,\ldots,P_N$.\\
Each partition is composed of $\#P_i=M_i$ domains $D_{i1},\ldots,D_{ij},\ldots,D_{iM_i}$.
Each domain has a number of items $I_{ij}$, a number of asynchronous ghosts to be transferred $A_{ij}$, 
and a number of local ghosts to be copied $B_{ij}$. We assume that we have an asynchronous relative
timing $\alpha$ and a local copy relative timing $\beta$.
So that the relative compute time of a domain is
$$
	\Theta_{ij} = I_{ij} + \alpha A_{ij} + \beta B_{ij}
$$
\end{document}
